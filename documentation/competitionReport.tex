\include{zLaTex-Preamble}

\usepackage{titlesec}
% \titleformat{\section}[block]{\centering\Large\bfseries\filcenter}{}{1em}{}
\titleformat{\subsection}[hang]{\centering\bfseries}{}{1em}{}

\title{
  Stratosphere \\ 
  \small{A multilevel data-driven forum for cloud computing}
} 

\author{
  Miller, Joshua \\
  Claxton, Spencer \\
  Mukora, Alice \\
  Griffis, Eric
}

\date{\small{18 June 2013}}

\newcommand\XXX[1]{{\color{red} #1}}

\begin{document}
\rhead{\fancyplain{}{Stratosphere Proposal}}

\maketitle

\begin{abstract}

  \XXX{Modern social networking tools are fine for some things, but
OSDC offers specific advantages to data researchers in finding and
forming relationships among peers that share research
interests. \ldots} In this proposal we discuss \emph{Stratosphere}, a
social component for OSDC that provides data researchers with new
opportunities to identify and connect to peers with similar interests.
Stratosphere is a collection of data-oriented tools that connects
cloud researchers with similar data consumption patterns. Our methods
for doing this include data access record aggregation and
presentation, peer authentication of user generated data, and making
users and their research visible to other users.

  %% unnecessary:
  %% 
  %% We are attempting to sell an idea here. We have no proof of concept
  %% save the precedent of other social networks and the benefits they offer.

\end{abstract}

\tableofcontents
\newpage
\begin{multicols*}{2}

\section{Introduction}
\label{sec:intro}

\XXX{Modern social networking tools are fine for some things, but OSDC
offers specific advantages to data researchers in finding and forming
relationships among peers that share research interests. \ldots}

By adding a social component to the OSDC experience, we aim to provide
data researchers with a unique opportunity to identify and connect to
peers with similar interests and data consumption habits. Beyond the
obvious benefits of conventional social networking tools like Facebook
for fostering collaboration, OSDC users also stand benefit from the
fact that their interests are encoded in their data usage patterns.

%% if you don't want to reference sections explicitly, rewrite
%% "Section \ref{...}" in the following paragraph with something
%% natural like "First, we ...'' and then disable section numbering.

This proposal proceeds as follows. Section \ref{sec:why} motivates the
Stratosphere design by clearly defining what we perceive to be the
weaknesses of popular social networks for fostering data intensive
research collaboration relationships in light of the unique
opportunities offered by OSDC. Section \ref{sec:what} outlines our
vision for the initial Stratosphere design. Section \ref{sec:how}
details the implementation mechanisms required to implement our
vision. Section \ref{sec:summary} summarizes this proposal concisely.

\subsection{Why Stratosphere}
\label{sec:why}

Stratosphere attempts to leverage implicit inter-connectivity between
scientists using the OSDC system. Currently, researchers are typically
connected to each other a priori, by institution or through
conferences or prior collaborations. The core concept of scientific
cloud computing in an open setting is the promotion of collaboration
and multifaceted approaches to data analysis. Users are currently cut
off from peer collaboration on research projects conducted via the
cloud. By distributing the responsibility of project documentation and
making users visible to one another, the OSDC would be promoting
direct interactions between researchers from different projects that
express interest in the same datasets. We suspect that connections
forged within the context of Stratosphere are more likely to lead to
successful collaborations because such connections are derived from
explicit matchings within the interest and problem domains.

\section{What is Stratosphere}
\label{sec:what}

Stratosphere is a data-driven social networking platform that connects
researchers in related groups and disciplines. Stratosphere is
composed of two layers: the user interaction layer and the data
interaction layer. The user interaction layer is a superstructure
designed with social networking in mind to create a social
interdisciplinary network centered around datasets. The data
interaction layer handles user access to the data. Stratosphere
provides a peer approval system for file permissions of user submitted
data and informs users of who is using which data and with what
intent.

\section{How Stratosphere Works}
\label{sec:how}

\XXX{Our implementation involves 

\subsection{A Data-driven Network}

The main infrastructure of Stratosphere is \emph{data driven}, which
means that user data interactions are logged in a metadata descriptor
attached to each data set. For example, when a user accesses a
particular dataset, rate counters specific to each individual user
could be incremented. Using these rates, we could present lists of the
most active users for each dataset to other users who express interest
in the same datasets. Furthermore, data frequently used by an
individual would be represented as specializations on the Stratosphere
front end. A metadata entry includes a username, the most recent
access time, and an access tally. Other relevant user-supplied
information like project descriptions and contact information would be
stored similarly.

\figH {metaRecording.png}{Process of recording user access metadata}{metarecord}{.5}       

\subsection{Peer Authentication}

This social-interactive system aids the moderation of user
permissions.  When users add data to the cloud they are able to assign
ownership to the data set, with the default being public
ownership. Publicly owned data sets require no permissions to access.
Should the creator of a data set request ownership, then he,she, or
selected other users take the responsibility of being an
administrative user.  Users are then granted access to the data set by
the administrative user.

\subsection{Visibility and Privacy}

The main goal of Stratosphere is to provide transparency through the
cloud. To do this we propose that users be given public profiles that
display the data they are working with and their contact information.
However, we are well aware that some actions need not be recorded in a
user's data access. For this reason, we propose the use of defaults
which lean towards full record keeping and visibility, with the option
to remove such visibility.

\subsection{User-to-Data Interactions}

The user-to-data interaction is the most challenging aspect of this
project. Required tools include implementations for
\begin{itemize}
\item file monitoring for user access records
\item peer authenticated permissions
\end{itemize}

\subsection{File Monitoring}

Possibilities for monitoring the frequency of data access include
tools such as \textit{inotify}, or direct patching of the file
mounting, or other proprietary/open software.  Preliminary testing of
accepted tools such as inotify seem improbable to use.  Starting the
service ran at approximately 16 GB/s for an 80GB section of the open
data. This appears to be non-scalable (~70 hours to index the entire
cloud). Therefore, we propose a patch to the mounting system such that
each new instance of a file access increments the metadata use
counter.  The main assumption is that the users who access the data
the most have the highest interest and experience with the particular
data set.

\subsection{Metadata Creation}

We propose to include metadata based on who accesses the data, how
often they access the data, and for what purpose they are accessing
the data.  In order to maintain privacy, the user is not required, but
encouraged to assign a purpose comment or project to each data set
accessed. This data set descriptor persists for each user until
changed.  When the data is accessed, it is the job of the file
mounting system to intercept and record the request.  We have provided
a proof of concept implementation (using XML metadata) of a file
system mount adapted from \textbf{fusedev} for this purpose.

\subsection{Permissions}

The task of creating peer authenticated permissions is a far more
challenging problem. Currently, the OSDC cloud operates on basic Unix
group permissions, and due to the binary grouping of open data and
protected data, this works for now. However, we would be forced to
store the entire combination of each user inclusion in each research
permission group. A file system with \textit{Dropbox}-style
permissions is certainly preferable.However, we currently have no
recommendation for how such a file system would be implemented so that
it is big-data scalable.

\subsection{User-to-User Interaction}

The main aspect of interactions between users in the Stratosphere is
the ability of the metadata to describe itself in terms of
users. Users are able to see what peers are doing with the data of
interest and are further able to get to a full research profile of the
users that are working with the same data. To make this possible we
propose the implementation of user profiles that are accessible from
and linked to the data that the user has accessed. The biggest
challenge here would be integrating such a social network with the
current OSDC interface (the Tukey portal).

\section{Summary}
\label{sec:summary}

We believe that, in light of the unusual opportunities made possible
by OSDC infrastructure, popular social networks are poorly positioned
to target opportunities for data intensive research to a comparable
degree. To tap this new resource we proposed Stratosphere, a social
networking platform that we believe rings true to the spirit of OSDC's
scientific cloud computing effort. Finally, to substantiate the
feasibility of our proposal, we outlined a specific implementation
strategy along with the details of various other practical matters,
such as what we believe to be the most significant sources
of \XXX{error and uncertainty.}

\XXX{Something witty, profound, clever, etc. to tie off the end.}

\end{multicols*}
\end{document}
