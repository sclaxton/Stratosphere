\documentclass[10pt]{article}
\usepackage[table]{xcolor}
\usepackage{fancyhdr}
\usepackage{amsmath} 
\usepackage[margin=3cm]{geometry}
\usepackage{graphicx} 
\usepackage{longtable}  
\usepackage{multicol} 
\usepackage{float}
\usepackage{xparse}
\usepackage{amssymb}
\newcommand{\degrees}{\ensuremath{^\circ}}
\usepackage{wrapfig}
\usepackage{parskip}
\usepackage{listings}
\usepackage{enumerate}
\newcommand{\csection}[1]{\section[#1]{\centering #1}}
\newcommand{\csubsection}[1]{\subsection[#1]{\centering #1}}


\newcommand{\code}[1]{\lstinline{#1}}
\renewcommand{\d}{\ensuremath{\operatorname{d}\!}}
\newcommand{\cin}[1]{{\scriptsize \lstinputlisting{#1}}}

\newcommand{\del}{\vec\nabla}
\newcommand{\p}{\partial}
\newcommand{\E}[1]{\times10^{#1}}
\newcommand{\eq}[1]{\begin{equation*}#1\end{equation*}}
\newcommand{\al}[1]{\begin{align*}#1\end{align*}}
\newcommand{\aln}[1]{\begin{align}#1\end{align}}
\newcommand{\sub}[1]{_\text{#1}}
% \newcommand{\qed}{\hfill \ensuremath{\Box}}
\newcommand{\vv}{\mathbf}
% \newcommand{\qed}{\hfill \ensuremath{\Box}~~~~~~~}

\newcommand{\dd}[2]{\frac{\d\,#1}{\d\,#2}}
\newcommand{\ddo}[1]{\frac{d}{d\,#1}}
\newcommand{\ddt}[1]{\frac{\d\,#1}{\d\,t}}
\newcommand{\dds}[2]{\frac{\d^2\,#1}{\d\,#2^2}}
\newcommand{\pd}[2]{\frac{\partial \,#1}{\partial \,#2}}
\newcommand{\spd}[2]{\frac{\partial^2 #1}{\partial #2^2}}
\newcommand{\lp}{\left(}
\newcommand{\rp}{\right)}

\newcommand{\lb}{\left[}
\newcommand{\rb}{\right]}


\usepackage{color}

\definecolor{dkgreen}{rgb}{0,0.6,0}
\definecolor{gray}{rgb}{0.5,0.5,0.5}
\definecolor{mauve}{rgb}{0.58,0,0.82}

\lstset{
  language=C,          
  % backgroundcolor=\color{gray}
  keywordstyle=\color{blue}\sffamily\bfseries,  
  commentstyle=\color{dkgreen}\itshape,    
  backgroundcolor=\color{Yellow!25},
  basicstyle=\sffamily,
  % basicstyle=\footnotesize,  
  numbers=left,              
  numberstyle=\footnotesize,      
  stepnumber=5,                   
  numbersep=5pt,                  
  backgroundcolor=\color{white},  
  showspaces=false,               
  showstringspaces=false,         
  showtabs=false,                 
  frame=single,           
  framerule=0.2pt,%expands outward 
  % rulecolor=\color{red},
  % framesep=3pt,%expands outward
  tabsize=4,          
  captionpos=b,           
  breaklines=true,        
  breakatwhitespace=false,    
  xleftmargin=3.2pt,
  xrightmargin=15pt,
}

\definecolor{lightblue}{rgb}{0.83,0.85,1.0}

\newcommand{\shadetable}[5]
{
  \begin{table}[H] 
    \begin{center}
      \caption{#3} \vspace{5pt}
      \rowcolors{1}{}{lightblue}
      \begin{tabular}{#1} \hline 
        #2 \\
        \hline
        #4 \\
        \hline                       
      \end{tabular}
      \label{#5}
    \end{center}
  \end{table}
}
% \shadetable{c|c}{column1 & column2}
% {Test Table 1}{  
%   a & b \\
%   c & d
% }{tab:test1}



\DeclareDocumentCommand\m{ m g g g g g g g g}
  {
    \left( 
      \begin{smallmatrix} 
        #1
        \IfNoValueF{#2}{\\#2}
        \IfNoValueF{#3}{\\#3}
        \IfNoValueF{#4}{\\#4}
        \IfNoValueF{#5}{\\#5}
        \IfNoValueF{#6}{\\#6}
        \IfNoValueF{#7}{\\#7}
        \IfNoValueF{#8}{\\#8}
        \IfNoValueF{#9}{\\#9}
      \end{smallmatrix} 
    \right)
}
% \m{row1}[{row2}][{...}{...}...]

\newcommand{\fig}[4]
{
  \begin{figure}[h] \centering \vspace{-10pt} 
    \includegraphics[width=#4\textwidth]{#1} \vspace{-10pt} 
    \caption{#2} \label{#3}
  \end{figure}
}
% \figH {FILENAME}{CAPTION}{LABEL}{WIDTH}       

\newcommand{\figH}[4]
{
  \begin{figure}[H] \centering \vspace{-10pt} 
    \includegraphics[width=#4\textwidth]{#1}\vspace{-10pt} 
    \caption{#2} \label{#3}
  \end{figure}
} 
% \figH {FILENAME}{CAPTION}{LABEL}{WIDTH}       


% \AtBeginDocument{%
  % \setlength\abovedisplayskip{0pt}
  % \setlength\belowdisplayskip{0pt}}

% ------------------------------------------------------------------------
%                               Page Style
% ------------------------------------------------------------------------

\pagestyle{fancy}
\renewcommand{\sectionmark}[1]{\markright{\thesection\ #1}}
\fancyhf{}

\lhead{\fancyplain{}{\rightmark }}
\cfoot{\fancyplain{}{\thepage}}


\usepackage{titlesec}
% \titleformat{\section}[block]{\centering\Large\bfseries\filcenter}{}{1em}{}
\titleformat{\subsection}[hang]{\centering\bfseries}{}{1em}{}

\title{
  Stratosphere \\ 
  \small{A multilevel data-driven social-network for cloud computing}
} 

\author{
  Miller, Joshua \\
  Claxton, Spencer \\
  Mukora, Alice \\
  Griffis, Eric
}

\date{\small{18 June 2013}}

\newcommand\XXX[1]{{\color{red} #1}}

\begin{document}
\rhead{\fancyplain{}{Stratosphere Proposal}}

\maketitle

\begin{abstract}

Conventional social networking tools are fine for some things, but
to our knowledge there has not yet been a social networking tool
driven by data and research for scientists. OSDC offers an
environment with big advantages to big data researchers. In our
opinion the OSDC offers a unique opportunity for scientists to form 
relationships among peers that share research interests centered
around the data analyzing power that it provides. In this proposal 
we discuss \emph{Stratosphere}, a social network extension of the current OSDC
console that provides data researchers with new opportunities to
identify and connect to peers with similar research interests.
Stratosphere is a collection of data-oriented tools that connects
cloud researchers with similar data consumption patterns. Our methods
for doing this include data access record aggregation and
presentation, peer authentication of user generated data, and making
users and their research visible to other users. Looking toward
the future, post-implementation of Stratosphere, we conclude by
proposing that Stratosphere could also facilitate a paradigm for
quantifying scientific collaboration in the domain of scientific
computing. 
 
\end{abstract}

\tableofcontents
\newpage
\begin{multicols*}{2}

\section{Introduction}
\label{sec:intro}

By adding a social component to the OSDC experience, we aim to provide
data researchers with a unique opportunity to identify and connect to
peers with similar interests and data consumption habits. Beyond the
obvious benefits of conventional social networking tools like Facebook
for fostering collaboration, OSDC users also stand benefit from the
fact that their interests are encoded in their data usage patterns.
\\
\\
This proposal proceeds as follows. First we motivate
the Stratosphere design by clearly defining what we perceive to be the
insufficiency of popular social networks for fostering data intensive
research collaboration relationships, in light of the unique
opportunities offered by the OSDC. We also here highlight what
Stratosphere adds to the OSDC in terms of how efficiently data is handled
and curated. We then proceed to outline our
vision for the initial Stratosphere design. Finally we end by
detailing the necessary mechanisms required to implement our
vision for Stratosphere. We conclude with a glance toward the
possibilities that Stratosphere offers OSDC in the future. 

\subsection{Why Stratosphere}
\label{sec:why}

Stratosphere attempts to leverage implicit inter-connectivity between
scientists using the OSDC system. Currently, researchers are typically
connected to each other a priori, by institution or through
conferences or prior collaborations. The core concept of scientific
cloud computing in an open setting is the promotion of collaboration
and multifaceted approaches to data analysis. Users are currently cut
off from peer collaboration on research projects conducted via the
cloud. By distributing the responsibility of project documentation and
making users visible to one another, the OSDC would be promoting
direct interactions between researchers from different projects that
express interest in the same datasets. We suspect that connections
forged within the context of the OSDC and Stratosphere are more likely to lead to
successful collaborations because such connections are derived from
explicit matching within the interest and problem domains.
\\
\\
We also foresee that by keeping track of the data usage of users,
Stratosphere will enable the OSDC administration to be more sensitive to the
data-oriented needs of its users. For example, Stratosphere would allow the
OSDC to have more insight into choosing what data to keep more or less
up to date on the cloud based on what data is more or less relevant to users.
Along the same lines, when peer-to-peer authentication is a reality on
the OSDC, Stratosphere provides a way for users to monitor who is
working with their data and how often. 

\section{What is Stratosphere}
\label{sec:what}

Stratosphere is a data-driven social networking platform that connects
researchers in related groups and disciplines. Stratosphere is
composed of two layers: the user interaction layer and the data
interaction layer. The user interaction layer is a superstructure
designed with social networking in mind to create a social
interdisciplinary network centered around datasets. The data
interaction layer handles user access to the data. Stratosphere
provides a peer approval system for file permissions of user submitted
data and informs users of who is using which data and with what
intent.
\vspace{10}
\figH{CDiagram_gray.png}{Stratosphere structure}{strat}{.5}


\section{How Stratosphere Works}
\label{sec:how}

\subsection{A Data-driven Network}

The main infrastructure of Stratosphere is \emph{data-driven}. This
mean that all interaction between users and the rest of the network is
mediated through the data. One way we propose to accomplish this is by logging
user-data interaction in a metadata descriptor attached to each data
set. For example, when a user accesses a particular dataset, 
rate counters specific to each individual user
could be incremented. Using these rates, we could present lists of the
most active users for each dataset to other users who express interest
in the same datasets. Furthermore, data frequently used by an
individual would be represented as specializations on that users
profile on Stratosphere front end. A metadata entry might 
include a username, the most recent access time, and an access tally. 
Other relevant user-supplied information like project descriptions 
and contact information would be stored similarly.


\figH {metaRecording.png}{Process of recording user access metadata}{metarecord}{.5}       

\subsection{Peer Authentication}

This social-interactive system would also provide a framework for
the moderation of user permissions.  When 
users add data to the cloud they are able to assign
ownership to the data set, with the default being public
ownership. Publicly owned data sets require no permissions to access.
Should the creator of a data set request ownership, then he,she, or
selected other users take the responsibility of being an
administrative user. Users are then granted access to the data set by
the administrative user.

\subsection{User Visibility and User-to-User Interaction}

The main goal of Stratosphere is to provide transparency through the
current OSDC system. To do this we propose that users be given public profiles that
display the data they are working with and their contact information.
However, we are well aware that some actions need not be recorded in a
user's data access. For this reason, we propose the use of defaults
which lean towards full record keeping and visibility, with the option
to remove such visibility. The primary way in which users will access
these public profiles will be, again, data-centered. That is, users
will have access to other users by seeing their the
data-access records in the metadata. Thus, the data becomes the
primary link between researchers. It is this ability of the metadata 
to describe the data in terms of users access that makes the
Stratosphere a truly data-driven social network. In this way, users are able to see 
directly what peers are doing with the data they themselves are
accessing. An implementation of user profiles would simply
entail an addition to the current OSDC web front end that displays user info pulled
from a database containing user activity statistics. 

\subsection{User-to-Data Interactions}

The user-to-data interaction is the most challenging aspect of this
project. Required tools include implementations for
\begin{itemize}
\item file monitoring for user access records
\item peer authenticated permissions
\end{itemize}

\subsection{File Monitoring}

Possibilities for monitoring the frequency of data access include
tools such as \emph{inotify}, or direct patching of the file
mounting, or other proprietary/open software.  Preliminary testing of
accepted tools such as inotify seem improbable to use.  Starting the
service ran at approximately 16 GB/s for an 80GB section of the open
data. This appears to be non-scalable (~70 hours to index the entire
cloud). Therefore, we propose a patch to the mounting system such that
each new instance of a file access increments the metadata use
counter.  The main assumption is that the users who access the data
the most have the highest interest and experience with the particular
data set.

\subsection{Metadata Creation}

We propose to include metadata based on who accesses the data, how
often they access the data, and for what purpose they are accessing
the data.  In order to maintain privacy, the user is not required, but
encouraged to assign a purpose comment or project name to each data set
accessed. This data set descriptor persists for each user until
changed.  When the data is accessed, it is the job of the file
mounting system to intercept and record the request. Our initial plan
for a prototype implementation of such a tool  was to mount a FUSE driven file
system to a WebDAV share on a web server. File access request would
then be intercepted and logged by the server. The user information
contained in the server log would then be dropped into the metadata
record. The main issue with this implementation is its scalability to
an OSDC system with many users accessing a large number of very large
datasets.        

\subsection{Permissions}

The task of creating peer authenticated permissions is a far more
challenging problem. Currently, the OSDC cloud operates on basic Unix
group permissions, and due to the binary grouping of open data and
protected data, this works for now. However, we would be forced to
store the entire combination of each user inclusion in each research
permission group. A file system with \emph{Dropbox}-style
permissions is certainly preferable. However, we currently have no
recommendation for how such a file system would be implemented so that
it is big-data scalable.


\section{Conclusion and the Future}
\label{sec:conclusion}

We believe that, in light of the unusual opportunities made possible
by the OSDC infrastructure, conventional social networks are poorly positioned
to target opportunities for data intensive research collaboration
between scientists. To tap this new resource we proposed Stratosphere, a social
networking platform that we believe rings true to the spirit of the OSDC's
scientific cloud computing effort. We envision that with Stratosphere, 
the OSDC will become a microcosm for quantifiable collaboration between
scientists using computational methods over big-data. That is, with
Stratosphere, we will be able to gather data on how scientists
interact with each other through the data that they are analyzing. We could
thereby quantify in some way the collaboration that takes place
between scientists, e.g. looking at the amount of analysis performed
on a dataset before and after implementation of Stratosphere on the OSDC.  

\end{multicols*}
\end{document}
